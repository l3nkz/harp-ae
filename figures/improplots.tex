\documentclass[9pt]{standalone}

\usepackage[english]{babel}
\usepackage[T1]{fontenc}
\usepackage[utf8]{inputenc}

\newcommand{\mean}{\textsc{Harp}}

\usepackage{tikz}
\usepackage{pgfplots}

\usetikzlibrary{colorbrewer}

\pgfdeclarelayer{bg}
\pgfdeclarelayer{fg}
\pgfsetlayers{bg,main,fg}

\pgfkeys{%
  /tikz/on layer/.code={
    \pgfonlayer{#1}\begingroup
    \aftergroup\endpgfonlayer
    \aftergroup\endgroup
  },
}

\makeatletter
\pgfdeclareshape{coordnode}{
  \inheritsavedanchors[from=rectangle]
  \inheritanchorborder[from=rectangle]
  \inheritanchor[from=rectangle]{center}
  \inheritanchor[from=rectangle]{base}
  \inheritanchor[from=rectangle]{north}
  \inheritanchor[from=rectangle]{north east}
  \inheritanchor[from=rectangle]{east}
  \inheritanchor[from=rectangle]{south east}
  \inheritanchor[from=rectangle]{south}
  \inheritanchor[from=rectangle]{south west}
  \inheritanchor[from=rectangle]{west}
  \inheritanchor[from=rectangle]{north west}
  \backgroundpath{
    %  store lower right in xa/ya and upper right in xb/yb
    \southwest \pgf@xa=\pgf@x \pgf@ya=\pgf@y
    \northeast \pgf@xb=\pgf@x \pgf@yb=\pgf@y
    \pgfpathmoveto{\pgfpoint{\pgf@xa}{\pgf@ya}}
    \pgfpathlineto{\pgfpoint{\pgf@xb}{\pgf@ya}}
 }
}

\pgfplotsset{
    /pgfplots/flexible xticklabels from table/.code n args={3}{%
        \pgfplotstableread[#3]{#1}\coordinate@table
        \pgfplotstablegetcolumn{#2}\of{\coordinate@table}\to\pgfplots@xticklabels
        \let\pgfplots@xticklabel=\pgfplots@user@ticklabel@list@x
    }
}
\makeatother

\definecolor{TimeColor}{HTML}{009ACD}
\definecolor{EnergyColor}{HTML}{43CD80}
% Uncomment if old colors
%\definecolor{TimeColor}{HTML}{D95F02}
%\definecolor{EnergyColor}{HTML}{1B9E77}

\pgfplotsset{
    invisible coord/.style={
        /tikz/coordinate,
    },
    coord connection anchor/.initial=south,
    coord connection direction/.initial=east,
    coord connection/.style n args={4}{
        inner xsep=1pt,
        inner ysep=1pt,
        xshift=-#3,
        yshift=-#4,
        anchor=#1 #2,
        append after command={
            [rounded corners=2pt](\tikzlastnode.#1) -- (\tikzlastnode.#1 #2) -- ++(#3, #4)
        }
    },
    visible coord/.style={
        coord connection/.code={
            \edef\coordanchor{\pgfkeysvalueof{/pgfplots/coord connection anchor}}
            \edef\coorddirection{\pgfkeysvalueof{/pgfplots/coord connection direction}}
            \def\north{north}
            \def\east{east}
            \ifx\coordanchor\north
                \def\yshift{2pt}
            \else
                \def\yshift{-2pt}
            \fi
            \ifx\coorddirection\east
                \def\xshift{2pt}
            \else
                \def\xshift{-2pt}
            \fi
            \edef\temp{\noexpand\pgfkeysalso{/pgfplots/coord connection={\coordanchor}{\coorddirection}{\xshift}{\yshift}}}
            \temp
        },
        coord connection,
    },
    nodes near coords above/.style={
        every node near coord/.append style={
            check for above values/.code={
                \begingroup
                % this group is merely to switch to FPU locally.
                % Might be unnecessary, but who knows.
                \pgfkeys{/pgf/fpu}
                \pgfmathparse{\pgfplotspointmeta<#1}
                \global\let\notabove=\pgfmathresult
                \endgroup
                %
                % simplifies debugging:
                %\show\above
                %
                \pgfmathfloatcreate{1}{1.0}{0}
                \let\ONE=\pgfmathresult
                \ifx\notabove\ONE
                    % AH: our condition 'y < #1' is met.
                    \pgfkeysalso{/pgfplots/invisible coord}
                \else
                    \pgfkeysalso{/pgfplots/visible coord}
                \fi
            },
            check for above values,
        }
    },
    nodes near coords below/.style={
        every node near coord/.append style={
            check for below values/.code={
                \begingroup
                % this group is merely to switch to FPU locally.
                % Might be unnecessary, but who knows.
                \pgfkeys{/pgf/fpu}
                \pgfmathparse{\pgfplotspointmeta>#1}
                \global\let\notbelow=\pgfmathresult
                \endgroup
                %
                % simplifies debugging:
                %\show\below
                %
                \pgfmathfloatcreate{1}{1.0}{0}
                \let\ONE=\pgfmathresult
                \ifx\notbelow\ONE
                    % AH: our condition 'y > #1' is met.
                    \pgfkeysalso{/pgfplots/invisible coord}
                \else
                    \pgfkeysalso{/pgfplots/visible coord}
                \fi
            },
            check for below values,
        },
    },
    nodes near coords outside/.style 2 args={
        every node near coord/.append style={
            check for outside values/.code={
                \begingroup
                \pgfkeys{/pgf/fpu}
                \pgfmathparse{\pgfplotspointmeta<#1}
                \global\let\notabove=\pgfmathresult
                \pgfmathparse{\pgfplotspointmeta>#2}
                \global\let\notbelow=\pgfmathresult
                \endgroup
                %
                \pgfmathfloatcreate{1}{1.0}{0}
                \let\ONE=\pgfmathresult
                \ifx\notabove\ONE
                    \ifx\notbelow\ONE
                        % Both conditions are met
                        \pgfkeysalso{/pgfplots/invisible coord}
                    \else
                        % Only below
                        \pgfkeysalso{/pgfplots/visible coord}
                    \fi
                \else
                    % Only above
                    \pgfkeysalso{/pgfplots/visible coord}
                \fi
            },
            check for outside values,
        },
    },
    x shift/.style 2 args={
        x filter/.code={\pgfmathparse{\pgfmathresult+((#1+.5-#2*.5)*.18)}}
    },
    table/filtered y error/.style n args={4}{
        % Filter error bar visualization by multiple arguments
        % #1 --> column name
        % #2 --> starting at which percentage should the bars be shown
        % #3 + #4 --> only show error bars when the corresponding y value is within this min/max range
        table/y error plus expr={\thisrow{#1} < #3 && \thisrow{#1} > #4 && \thisrow{err_#1_max} > (#2/100) ? \thisrow{err_#1_max} : nan},
        table/y error minus expr={\thisrow{#1} < #3 && \thisrow{#1} > #4 && \thisrow{err_#1_min} > (#2/100) ? \thisrow{err_#1_min} : nan}
    },
}

\pgfplotsset{
    improplot/.style={
        only marks,
        ylabel=Improvement Factor,
        ymin=0.34,
        ymax=6,
        restrict y to domain*=-0.375:0.75,
        flexible xticklabels from table={\base/rel/cfs.csv}{scenario}{col sep=comma},
        xtick=data,
        xticklabel style={rotate=20, font=\tiny, anchor=north east, yshift=3pt, xshift=5pt},
        every major x tick/.append style={opacity=0},
        ytick={0.1,0.2,0.6,2,4},
        yticklabels={0.1,0.2,0.6,2,4},
        yticklabel style={font=\tiny},
        y label style={at={(axis description cs:-0.04,.5)},anchor=south},
        extra y ticks={1},
        extra y tick labels={\textit{CFS}},
        extra y tick style={grid style={densely dotted, black}},
        grid=major,
        axis x line=bottom,
        x axis line style={draw opacity=0, -},
        axis y line=left,
        y axis line style={{Stealth[round]}-{Stealth[round]}},
        enlarge x limits={abs=0.4},
        enlarge y limits={abs=0.3},
        clip=false,
        error bars/y dir=both,
        error bars/y explicit relative,
        error bars/error bar style=solid
    },
    energy plot/.style={
        EnergyColor,
        coord connection anchor=south,
        every mark/.append style={fill=EnergyColor},
        mark size=1.8pt,
        visualization depends on=rawy\as\rawy,
        nodes near coords={\pgfmathprintnumber[fixed]{\rawy}},
        nodes near coords style={font=\tiny, text=black},
        nodes near coords outside={0.7}{-0.35},
        table/y={energy},
        table/filtered y error={energy}{5}{5}{0.45},
        table/col sep=comma,
    },
    time plot/.style={
        TimeColor,
        coord connection anchor=north,
        every mark/.append style={fill=TimeColor},
        mark size=1.8pt,
        visualization depends on=rawy\as\rawy,
        nodes near coords={\pgfmathprintnumber[fixed]{\rawy}},
        nodes near coords style={font=\tiny, text=black},
        nodes near coords outside={0.7}{-0.35},
        table/y={time},
        table/filtered y error={time}{5}{5}{0.45},
        table/col sep=comma,
    },
    runtime plot/.style={
        no markers,
        visualization depends on={\thisrow{time}\as\mylabel},
        nodes near coords={\pgfmathprintnumber[fixed]{\mylabel}\,s},
        nodes near coords style={font=\tiny, text=black, draw=gray!30, fill=gray!30, rounded corners=2pt, inner sep=.8pt},
        table/x expr=\coordindex,
        table/col sep=comma,
    }
}

\begin{document}

% Raptor Lake figure
\begin{tikzpicture}
\def\base{data/raptor/single}
\def\modes{td/diamond/east, tetris/oplus/east, tetris-offline/*/west, tetris-noscaling/o/east}

%Our table for the positions of the CFS runtimes
\pgfplotstableread[header=has colnames, col sep=comma]{\base/raw/cfs.csv}\cfsruntime
\pgfplotstableread[header=has colnames]{
pos
0.3
0.3
0.3
6
0.3
0.3
0.3
0.3
3
0.3
0.3
0.3
0.3
0.3
0.3
6
0.3
}\runtimeypos
\pgfplotstablecreatecol[copy column from table={\runtimeypos}{pos}] {ypos} {\cfsruntime}

\begin{axis} [
    improplot,
    width=\textwidth,
    height=4.3cm,
    ymode=log,
    log basis y=10,
    name=single
]
    % Create legend entries explicitly
    \foreach \mode/\mark/\direction in \modes {
        \edef\temp{
            \noexpand\addlegendimage{only marks, black, mark=\mark}
            \noexpand\label{leg_rap_\mode}
        }
        \temp
    }
    \addlegendimage{only marks, EnergyColor, mark=square*}\label{leg_rap_energy}
    \addlegendimage{only marks, TimeColor, mark=square*}\label{leg_rap_time}

    \addplot+ [
        no markers,
    ] table [x expr=\coordindex, y={energy}, col sep=comma] {\base/rel/cfs.csv};

    \foreach \mode/\mark/\direction [count=\i from 0] in \modes {
        \edef\temp{%
            \noexpand\addplot+ [energy plot,
                mark=\mark,
                coord connection direction=\direction,
                x shift={\i}{4}]
            table[x expr=\noexpand\coordindex] {\base/rel/\mode.csv};

            \noexpand\addplot+ [time plot,
                mark=\mark,
                coord connection direction=\direction,
                x shift={\i}{4}]
            table[x expr=\noexpand\coordindex] {\base/rel/\mode.csv};%
        }
        \temp
    }

    \addplot+ [runtime plot] table[y=ypos] {\cfsruntime};
\end{axis}

\def\base{data/raptor/multi}
\def\modes{td/diamond/east, tetris/oplus/east, tetris-offline/*/west, tetris-noscaling/o/east}

%Our table for the positions of the CFS runtimes
\pgfplotstableread[header=has colnames, col sep=comma]{\base/raw/cfs.csv}\cfsruntime
\pgfplotstableread[header=has colnames]{
pos
0.3
0.3
0.3
0.3
6
0.3
0.3
0.3
0.3
0.3
6
6
6
6
0.3
0.3
6
6
}\runtimeypos
\pgfplotstablecreatecol[copy column from table={\runtimeypos}{pos}] {ypos} {\cfsruntime}

\begin{axis} [
    improplot,
    width=\textwidth,
    height=4.3cm,
    ymode=log,
    log basis y=10,
    name=multi,
    anchor=north,
    at=(single.south),
    yshift=-.8cm,
]
    % Add the cfs plot (without drawing it for in order to set the scene
    \addplot+ [
        no markers,
    ] table [x expr=\coordindex, y={energy}, col sep=comma] {\base/rel/cfs.csv};

    \foreach \mode/\mark/\direction [count=\i from 0] in \modes {
        \edef\temp{%
            \noexpand\addplot+ [energy plot,
                mark=\mark,
                coord connection direction=\direction,
                x shift={\i}{4}]
            table[x expr=\noexpand\coordindex] {\base/rel/\mode.csv};

            \noexpand\addplot+ [time plot,
                mark=\mark,
                coord connection direction=\direction,
                x shift={\i}{4}]
            table[x expr=\noexpand\coordindex] {\base/rel/\mode.csv};%
        }
        \temp
    }

    \addplot+ [runtime plot] table[y=ypos] {\cfsruntime};
\end{axis}

% Draw the legend explicitly
\begin{pgfonlayer}{fg}
\node[font=\tiny, anchor=north] (modes) at (rel axis cs: 0.42,1.5) {
\begin{tabular}{ll}
\ref{leg_rap_td} ITD &
\ref{leg_rap_tetris-offline} \mean{} (Offline) \\
\ref{leg_rap_tetris} \mean{} &
\ref{leg_rap_tetris-noscaling} \mean{} (No Scaling)
\end{tabular}
};
\node[left=0pt of modes] (modes_text) {\textbf{\footnotesize{Modes}}};

\node[right=.6cm of modes] (metrics_text) {\textbf{\footnotesize{Metrics}}};
\node[font=\tiny, right=0pt of metrics_text] (metrics) {
\begin{tabular}{ll}
\ref{leg_rap_time} Makespan &
\ref{leg_rap_energy} Energy
\end{tabular}
};
\end{pgfonlayer}
\node[
    fit=(modes_text)(metrics),
    draw,
    rounded corners=3pt,
    fill=white,
    inner sep=.5pt
    ] {};

\end{tikzpicture}

% Odroid figure
\begin{tikzpicture}
\def\base{data/odroid/single}
\def\modes{tetris/*/east}

%Our table for the positions of the CFS runtimes
\pgfplotstableread[header=has colnames, col sep=comma]{\base/raw/cfs.csv}\cfsruntime
\pgfplotstableread[header=has colnames]{
pos
6
0.3
6
0.3
6
0.3
6
0.3
6
0.3
3.6
0.3
3.6
}\runtimeypos
\pgfplotstablecreatecol[copy column from table={\runtimeypos}{pos}] {ypos} {\cfsruntime}

\begin{axis} [
    improplot,
    width=.55\textwidth,
    height=4.3cm,
    y label style={at={(axis description cs:-0.06,.5)},anchor=south},
    ymode=log,
    log basis y=10,
    name=single,
    % We need to adjust some values from the original plot style
    enlarge x limits={abs=0.7},
    extra y tick labels={\textit{EAS}},
]
    % Create legend entries explicitly
    \foreach \mode/\mark in \modes {
        \edef\temp{
            \noexpand\addlegendimage{only marks, black, mark=\mark}
            \noexpand\label{leg_odr_\mode}
        }
        \temp
    }
    \addlegendimage{only marks, EnergyColor, mark=square*}\label{leg_odr_energy}
    \addlegendimage{only marks, TimeColor, mark=square*}\label{leg_odr_time}

    \addplot+ [
        no markers,
    ] table [x expr=\coordindex, y={energy}, col sep=comma] {\base/rel/cfs.csv};

    \foreach \mode/\mark/\direction [count=\i from 0] in \modes {
        \edef\temp{%
            \noexpand\addplot+ [energy plot,
                mark=\mark,
                coord connection direction=\direction]
            table[x expr=\noexpand\coordindex] {\base/rel/\mode.csv};

            \noexpand\addplot+ [time plot,
                mark=\mark,
                coord connection direction=\direction]
            table[x expr=\noexpand\coordindex] {\base/rel/\mode.csv};%
        }
        \temp
    }

    \addplot+ [runtime plot] table[y=ypos] {\cfsruntime};
\end{axis}

\def\base{data/odroid/multi}

%Our table for the positions of the CFS runtimes
\pgfplotstableread[header=has colnames, col sep=comma]{\base/raw/cfs.csv}\cfsruntime
\pgfplotstableread[header=has colnames]{
pos
3.6
0.3
3.6
0.3
3.6
0.3
3.6
0.3
6
0.3
6
0.3
6
0.3
}\runtimeypos
\pgfplotstablecreatecol[copy column from table={\runtimeypos}{pos}] {ypos} {\cfsruntime}

\begin{axis} [
    improplot,
    width=.55\textwidth,
    height=4.3cm,
    ymode=log,
    log basis y=10,
    name=multi,
    anchor=west,
    at=(single.east),
    xshift=1em,
    % We need to adjust some values from the original plot style
    enlarge x limits={abs=0.6},
    y axis line style={draw opacity=0, -},
    ylabel={},
    extra y tick style={text opacity=0},
    yticklabel style={text opacity=0},
    every major y tick/.append style={opacity=0},
]
    % Add the cfs plot (without drawing it for in order to set the scene
    \addplot+ [
        no markers,
    ] table [x expr=\coordindex, y={energy}, col sep=comma] {\base/rel/cfs.csv};

    \foreach \mode/\mark/\direction [count=\i from 0] in \modes {
        \edef\temp{%
            \noexpand\addplot+ [energy plot,
                mark=\mark,
                coord connection direction=\direction]
            table[x expr=\noexpand\coordindex] {\base/rel/\mode.csv};

            \noexpand\addplot+ [time plot,
                mark=\mark,
                coord connection direction=\direction]
            table[x expr=\noexpand\coordindex] {\base/rel/\mode.csv};%
        }
        \temp
    }

    \addplot+ [runtime plot] table[y=ypos] {\cfsruntime};
\end{axis}

%\draw[thick] ($(single.south east) + (.5em,0)$) -- ($(single.north east) + (.5em, -1em)$);

% Draw the legend explicitly
\begin{pgfonlayer}{fg}
\node[fill=white, font=\tiny, anchor=north] (modes) at (rel axis cs: 0.95,1.5) {
\begin{tabular}{l}
\ref{leg_odr_tetris} \mean{} (Offline)
\end{tabular}
};
\node[left=0pt of modes] (modes_text) {\textbf{\footnotesize{Modes}}};

\node[right=.6cm of modes] (metrics_text) {\textbf{\footnotesize{Metrics}}};
\node[font=\tiny, right=0pt of metrics_text] (metrics) {
\begin{tabular}{ll}
\ref{leg_odr_time} Makespan &
\ref{leg_odr_energy} Energy 
\end{tabular}
};
\end{pgfonlayer}
\node[
    fit=(modes_text)(metrics), 
    draw,
    rounded corners=3pt,
    fill=white,
    inner sep=.5pt,
    ] {};
\end{tikzpicture}
\end{document}
